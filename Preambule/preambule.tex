Dans ce manuscrit, vous trouverez l'ensemble des résultats de mes travaux de thèse.
Avant de les présenter, je vais ici introduire le contexte dans lequel j'ai effectué ces travaux.
Je suis issu d'une formation en informatique fondamentale de l'université Lille 1 et spécialisé en algorithmique par un master de cette même université.
C'est dans le cadre de mon parcours universitaire que j'ai eu l'occasion de rencontrer Maude Pupin, mon actuelle directrice de thèse, et Laurent Noé co-encadrant.
Ils m'ont accueilli plusieurs fois en projets de master et stages, puis en thèse au sein de l'équipe Bonsai dirigée par Hélène Touzet, équipe de bioinformatique commune au laboratoire CRIStAL et à l'INRIA.

L'équipe Bonsai est une équipe de bioinformatique orientée algorithmique.
Plus précisément, l'équipe est orientée algorithmique des séquences, qu'elles soient ARN, ADN, de gènes ou moléculaires.
Ma thématique de recherche est, elle, orientée structures moléculaires pour des molécules appelées peptides non ribosomiques.
Au cours de mes 3 années de thèse j'ai interagi à de nombreuses reprises avec d'autres thématiques de l'équipe, mais deux discussions ont particulièrement mené à des collaborations hors de mon cadre de recherche usuel.

\begin{figure}[!ht]
  \begin{center}
    \includegraphics[width=350px]{Figures/preambule/pierre.png}
    \caption{\label{pierre}Contraction d'un graphe de read pour extraire les clusters à assembler ensuite.
    Ces données sont issues d'un métagénome simple à deux espèces.}
  \end{center}
\end{figure}

Pierre Péricard est également doctorant de l'équipe et travaille sur un pipeline d'assemblage de marqueurs conservés dans des données métagénomiques.
Durant le déroulement algorithmique du programme, il est amené à construire un graphe de reads présents dans l'échantillon où deux reads sont liés lorsqu'ils sont partiellement chevauchants (voir figure \ref{pierre}).
Il souhaitait pouvoir extraire les composantes ``linéaires'' pour effectuer des assemblages sur ces sous-jeux de read.
J'ai créé pour lui un programme qui permet de réduire ce graphe avec une épaisseur en un graphe filiforme facilement découpable au niveau des arêtes faibles (en rouge sur la figure) et des n\oe{}uds d'arité supérieure à 2.
Cet utilitaire permet la contraction des n\oe{}uds proches via des distances calculées par l'algorithme de Dijkstra, ainsi que l'utilisation de plusieurs filtres discriminant les reads ne contribuant pas à l'homogénéité du graphe.
Le pipeline complet sera prochainement publié par Pierre.

En début d'année, Hélène Touzet a publié un article contenant une méthode de programmation dynamique permettant le comptage des voisins d'un mot avec k erreurs.
Cette méthode utilise le croisement de deux automates dont l'un est l'automate universel déterministe de Levenshtein.
Le fait que cet automate universel ne dépende pas du mot pour lequel on recherche le voisinage permet de le générer une seule et unique fois pour un k donné, pour n'avoir plus qu'à le croiser avec le second automate lors de chaque exécution.
Depuis cette publication, je me suis proposé pour travailler avec Hélène sur l'implémentation de la méthode puis sur l'analyse de la minimalité de l'automate universel déterministe de Levenshtein que nous avons généré.
Ces travaux sont en cours et feront l'objet d'une publication prochainement.

~~

Mon sujet de thèse m'a également amené à collaborer avec de nombreuses personnes locales mais également d'équipes distantes.
Localement, j'ai travaillé avec plusieurs membres de l'Institut Charles Viollette, institut également hébergé par l'université Lille 1.
J'ai notamment beaucoup travaillé avec Valérie Leclère sur les aspects de biologie de mon sujet et Mickaël Chevalier concernant les aspects de spectrométrie de masse.

À l'extérieur, j'ai principalement travaillé avec Tilmann Weber et son équipe hébergée au Danemark par la Novo Nordist Foundation.
Ses travaux sur l'outil antiSMASH en font un collaborateur de longue durée au sein de la thématique des peptides non ribosomiques sur l'université Lille 1.
C'est d'ailleurs cette collaboration entre son équipe et la nôtre qui a permis l'organisation, à Lille et par deux fois (en 2013 et 2015), d'un workshop sur la thématique des outils bioinformatiques pour l'analyse des peptides non ribosomiques et des polykétides.
Cette collaboration de longue date m'a également permis d'être accueilli durant un mois au Danemark dans son équipe pour démarrer, avec son équipe, le développement d'une nouvelle application dont nous parlerons dans la partie \ref{bio_synth} de ce manuscrit.

Depuis maintenant plus d'un an, l'équipe autour de Norine collabore également avec l'équipe de recherche Suisse ``Proteome Informatics'' dirigée par Frédérique Lisacek, dans le but de créer un pipeline d'identification de NRP via spectrométrie de masse.
Deux doctorants ont été recrutés à Lille (Mickaël Chevalier) et Genève (Emma Ricart).
Durant ma thèse, j'ai collaboré avec Emma pour son utilisation de Smiles2Monomers (logiciel développé durant ma thèse) afin de valider les structures biologiques de ses peptides non ribosomiques.

~~

Globalement, mon contexte de travail a été particulièrement agréable et m'a permis de m'ouvrir l'esprit à la recherche mais également à la vie universitaire en général.
Les 3 ans et 6 mois d'enseignement en informatique et les différentes responsabilités universitaires, telles que la création et gestion d'un groupe d’entraînement algorithmique avec Thibault Raffaillac, m'ont préparé à l'éventualité de devenir un jour maître de conférences ou chercheur.



























