Comme nous l'avons vu dès le premier chapitre, avant ce travail, il n'existait aucun moyen à la fois rapide et précis pour obtenir des annotations biologiques de peptides non ribosomiques.
Les annotations étaient soit rapides mais issues de prédictions, soit précises car issues de données expérimentales (générées par exemple par spectrométrie), mais annotées à la main (section \ref{bioanalyse}).
Notre approche propose de venir compléter le processus de découverte de structures par spectrométrie en proposant smiles2monomers comme logiciel de création automatique des annotations biologiques (section \ref{problems}).
s2m a été développé dans l'optique d'être rapide, le plus précis possible dans l'annotation et non spécifique aux NRP.
Le logiciel effectue une annotation d'un polymère à partir d'une liste de monomères candidats.

Pour construire s2m, nous avons découpé le problème d'annotation en deux phases d'analyse informatique.
Dans une première étape, nous recherchons individuellement au sein du peptide chacun des monomères définis comme candidats par l'utilisateur (section \ref{subproblems}).
Les monomères ne sont pas recherchés tels quels mais dérivés en famille de résidus (section \ref{families}) pour pouvoir utiliser des techniques exactes de recherche de sous-graphe (section \ref{isomorphisme_p}).
Nous avons ensuite créé une structure d'indexation de ces résidus pour minimiser le temps d'exécution lors de la recherche (section \ref{index_p}).
Cette optimisation nous permet d'obtenir jusqu'à 35\% de gain en temps (section \ref{k_res}) lors des phases de recherche sans erreur, par rapport à l'état de l'art.
Certains résidus n'étant pas trouvables par une recherche stricte, nous avons également proposé une recherche \textit{light} beaucoup plus lente (section \ref{light_p}). Nous utilisons ce type de recherche uniquement en cas d'impossibilité de trouver une annotation complète à la suite d'une exécution stricte.

Dans une seconde étape, nous effectuons un pavage non chevauchant des résidus qui ont été trouvés lors de la première phase (section \ref{pavage_}).
Nous avons proposé une méthode exacte de pavage utilisant des résolutions par programmation linéaire (section \ref{MIP_p}) ou par une heuristique gloutonne localement raffinée (section \ref{TM_p}).
Comme nous nous y attendions, le pavage exact est bien plus lent que le pavage heuristique, mais nous avons également prouvé que, dans le contexte de son utilisation, il pouvait donner de moins bons résultats que le pavage heuristique.

Globalement, s2m est un logiciel rapide et efficace (section \ref{temps_calcul}), que nous avons utilisé comme contrôle qualité, ainsi que pour l'analyse de grands jeux de données.
Avec un temps d'exécution moyen de 18~ms par annotation sur les jeux de données test (Norine et CCD), le logiciel est extrêmement rapide et peut analyser des centaines de milliers de molécules si nécessaire (section \ref{res_analyse}).
De plus, avec 253 sur 342 annotations retrouvées à 100\%, une sensibilité de 0.9 sur les données issues de Norine, 318 annotations trouvées sur 378 pour CCD et une sensibilité 0.97 sur ces mêmes données, les résultats sont d'excellente qualité.
Enfin, comme nous l'avons vu durant l'analyse de ces résultats, ces chiffres sont une minoration de la qualité du logiciel car ils ne prennent pas en compte les erreurs découvertes au sein des jeux de données.

Les découvertes de diverses erreurs au sein des bases de données test nous ont démontré l'efficacité du logiciel et ont permis, via les corrections, d'augmenter la qualité globale des données de Norine (section \ref{cont_nor}).
De plus, en ajoutant s2m dans le pipeline d'ajout de nouvelles entrées dans Norine, nous nous sommes prémunis de futures erreurs.
Enfin, grâce à s2m, nous avons pu analyser en moins d'une journée, des centaines de milliers de peptides issus de bases généralistes pour trouver de potentiels NRP candidats (sections \ref{nor_fill} et \ref{nor_auto}).
s2m a généré de très importantes quantités de données qui vont désormais mener à l'entrée de nouvelles entrées de qualité dans Norine.


~~


Actuellement, ce travail ouvre deux perspectives à court terme.
Tout d'abord, en fin de manuscrit (section \ref{bio_synth}), j'ai présenté le début de création d'un outil qui a pour vocation d'aider la recherche en biologie de synthèse des NRP.
Pour le moment, cet outil ne peut être testé à cause de la faible quantité des données à notre disposition.
De plus, il est peu probable que le simple modèle présenté nous permette de prédire tout type de NRPS correctement.
Cependant intégrer d'autres informations que les domaines A dans le modèle permettrait sans doute des prédictions précises mais également vérifierait et approfondirait nos connaissances des structures, des domaines et des liaisons inter-domaines NRPS.
De plus la constitution d'une base de données (ou d'un ensemble de bases) liant toutes les étapes de la construction d'un NRP serait une ressource précieuse en soit.
Ces données amorceraient très certainement des campagnes de recherches de nouvelles NRPS par apprentissage.

La seconde perspective à court terme est la diffusion de s2m.
Nous avons identifié que plusieurs bases de données comme CCD~\cite{rahman_small_2009} ou le consortium HELM~\cite{_helm_????} souhaitant maintenir en permanence leurs polymères sous les formes à la fois atomiques et monomériques.
Sur son site web, HELM souligne particulièrement qu'il est difficile de trouver les structures monomériques dans les publications et bases de données publiques.
L'un des défis qu'ils référencent sur leur site est la transformation automatique des formats chimiques vers les formats monomériques.
s2m est parfaitement capable de relever ce défi puisqu'il a été conçu pour pouvoir fonctionner avec n'importe quel type de polymères.

À plus long terme, ce travail pourra être continué pour accroître à la fois les connaissances théoriques sur les NRP et NRPS tout en développant l'interaction avec des pharmacologues pour les aspects traitements antibactériens et antifongiques.
Notre avenir antibiotique est pour le moment limité et j'espère que ce travail pourra, au moins en partie, aider à la découverte de nouvelles substances d'intérêt pour la santé.

























