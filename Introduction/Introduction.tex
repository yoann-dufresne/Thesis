Les peptides non ribosomiques (NRP en anglais) sont des molécules synthétisées par des bactéries et champignons microscopiques.
Ces molécules sont d'une importance capitale pour ces organismes car elles sont souvent utilisées comme mécanisme de défense contre d'autres micro-organismes.
Nous, Humains, nous intéressons à ces molécules car elles sont une source importante de nouvelles molécules pour la pharmacologie.
En particulier, une très grande partie des NRP découverts à ce jour ont des propriétés antibiotiques.
La célèbre pénicilline, découverte par Alexander Fleming au début du siècle dernier, est une molécule issue d'une transformation d'un précurseur NRP.

Comme leur nom l'indique, les peptides non ribosomiques ne sont pas produits par la voie de synthèse classique des protéines (utilisant le ribosome).
Cette voie de synthèse alternative autorise la cellule à créer des molécules de formes et de compositions inhabituelles.
Les molécules sont assemblées à partir d'éléments de base appelés monomères.
Il existe actuellement plus de 500 monomères différents répertoriés comme étant inclus au sein de NRP.
Ce sont les compositions inhabituelles couplées aux formes particulières qui confèrent leur diversité d'activité et efficacité aux NRP.
Connaître les compositions des NRP est d'une importance cruciale car c'est ce qui nous permet de relier un composé à sa voie de synthèse et aussi de prédire l'activité que peut avoir cette molécule.

Les structures NRP sont découvertes via deux moyens.
D'un côté, il existe des logiciels d'analyse d'ADN qui permettent de détecter les gènes menant à la création de ces molécules.
Ces logiciels prédisent, à partir de l'ADN, les différents constituants potentiels des NRP produits.
Cependant, en l'état actuel des connaissances, ces techniques ne peuvent pas complètement prédire les formes et compositions complètes des peptides qui seront synthétisés.
D'un autre côté, il est possible de découvrir expérimentalement des NRP en analysant les composés produits par les organismes.
Ce processus permet d'obtenir, par spectrométrie entre autre, les structures chimiques des molécules.
Cependant, pour obtenir les informations biologiques (les monomères présents dans le peptide), il est souvent nécessaire d'effectuer une annotation manuelle.
Là où la première méthode de découverte donne rapidement de nombreuses annotations incomplètes, la seconde méthode donne des annotations exactes mais avec un bien plus faible débit du fait du traitement manuel.
Ma thèse et tout ce que je vais présenter dans ce manuscrit, s'articule autour de l'obtention rapide et exacte des annotations biologiques et de leur utilisation.

~~

Le manuscrit sera découpé en trois chapitres.
Le premier chapitre parlera de biologie et sera découpé en deux sections.
La section \ref{bio_NRP} abordera dans le détail les voies de synthèse des peptides non ribosomiques et la section \ref{bioanalyse} sera une vue globale des différents logiciels de découverte et stockage d'annotations des NRP.

Le second chapitre abordera la génération automatique d'annotations biologiques de NRP et sera découpé en 5 sections.
Après une courte introduction, la section \ref{problems} définira précisément la problématique informatique liée à l'annotation automatique.
Elle sera suivie en section \ref{SI_MCS} par un état de l'art de solutions algorithmiques pour résoudre ces problèmes.
Puis au sein de la section \ref{algos_s2m} nous expliquerons en détail notre construction algorithmique pour produire rapidement et automatiquement des annotations biologiques.
Enfin, en section \ref{res_int}, nous effectuerons de nombreux tests pour valider le modèle algorithmique proposé précédemment.

Le dernier chapitre, découpé en 3 sections sera, quant à lui, consacré à l'utilisation des résultats produits par le logiciel d'annotation.
La section \ref{nor_3} détaillera le fonctionnement de la base de donnée Norine, base au sein de laquelle nous avons intégré les annotations générées.
Puis nous expliquerons en détail en section \ref{cont_nor}, comment notre logiciel a contribué à l'amélioration de la qualité et de la quantité d'informations dans Norine.
Enfin, en section \ref{bio_synth}, nous aborderons les perspectives d'utilisation des annotations biologiques pour avancer progressivement vers la création de NRP de synthèse.